\section{Bumpy black hole spacetimes}\label{sec:spacetimes}
A metric that represents a rotating black hole is required to be stationary and axisymmetric (i.e. there exists a timelike Killing vector $\partial/\partial t$ and a spacelike Killing vector $\partial/\partial\phi$), invariant under simultaneous inversion of the $\phi$ and $t$ coordinates, and reflection symmetric about the equatorial plane. A sufficiently general metric which captures all these properties is given by \citep{Chandrasekhar:579245}
\begin{equation}\label{eq:generalmetric} \textrm{d}s^{2}=g_{tt}\,\textrm{d}t^{2} + g_{t\phi}\,\textrm{d}t\,\textrm{d}\phi + g_{rr}\,\textrm{d}r^{2} + g_{\theta\theta}\,\textrm{d}\theta^{2} + g_{\phi\phi}\,\textrm{d}\phi^{2}\, , \end{equation}
where the metric coefficients depend only on the radial and polar coordinates $r$ and $\theta$. In fact Eq.\ \ref{eq:generalmetric} retains considerable degree of gauge freedom, however, it is sufficient for our present purpose. 

As our interest here is in testing the hypothesis that the metric is the Kerr solution, and it is known that the Kerr solution is an excellent description in the weak field, it is natural to expand the metric as
\begin{equation} g_{\mu \nu}=g^{\textrm{Kerr}}_{\mu \nu}+\epsilon h_{\mu \nu} \, ,\end{equation}
where $\epsilon \ll 1$ and $h_{\mu\nu}\rightarrow 0$ as $r\rightarrow\infty$. The background,  Kerr solution is given by,
\begin{eqnarray} &g^{\textrm{Kerr}}_{tt}=-\left( 1-\frac{2Mr}{\rho^{2}} \right)\, , \;  g^{\textrm{Kerr}}_{t\phi}=\frac{-2M^{2}ar}{\rho^{2}}\sin ^{2} \theta \, ,\nonumber \\
&g^{\textrm{Kerr}}_{rr}=\frac{\rho^{2}}{\Delta}\, , \; g^{\textrm{Kerr}}_{\theta \theta}=\rho^{2} \, , \; g^{\textrm{Kerr}}_{\phi \phi}=\frac{\Sigma^{2}}{\rho^{2}}\sin ^{2} \theta \, , \end{eqnarray}
for a BH with a mass $M$ and dimensionless spin $a$, in which
\begin{eqnarray} &\rho^{2}=r^{2}+a^{2}M^{2}\cos ^{2} \theta \, , \; \Delta = r^{2}\left( 1-\frac{2M}{r} \right) +M^{2}a^{2} \, ,\\
&\textrm{and}\quad \Sigma^{2} = \left( r^{2}+M^{2}a^{2} \right)^{2} - M^{2}a^{2}\Delta \sin ^{2} \theta \, \nonumber .\end{eqnarray}
From here on dimensionless units with $M=1$ are used.

Geodesic motion with four-momentum $p^{\mu}$ in a metric with the symmetries outlined thus far would posses three constants of motion; energy ${E=-p^{\mu}(\partial/\partial t)_{\mu}}$, z-component of angular momentum ${L=p^{\mu}(\partial/\partial \phi)_{\mu}}$, and particle rest mass ${m^{2}=-p^{\mu}p_{\mu}}$. However the Kerr solution also possess an additional constant of motion related to the existence of a second rank Killing tensor, ${C=p^{\mu}p^{\nu}\xi_{\mu\nu}}$. In place of $C$ the Cater constant is often defined as ${Q=C-(L-aE)^{2}}$ \citep{PhysRev.174.1559}. The existence of the addition constant of motion ensures that geodesic motion in the spacetime is separable and tri-periodic (i.e.\ with a well defined frequency associated with motion in each of the $r$, $\theta$ and $\phi$ coordinates). Although it is not clear that it should be required of any alternative to the Kerr solution to possess these properties, they are sufficiently appealing that it is worth considering the possibility carefully.

The most general form of metric perturbation, $h_{\mu\nu}$, that could be added to the Kerr solution such that the resultant metric also admitted a second rank Killing tensor and hence a Carter-like constant (at least to ${\cal{O}} (\epsilon ^{2})$ in the perturbation) was considered by \cite{2011PhRvD..83j4027V}. The metric perturbation was also required to tend to zero in the limit $r\rightarrow\infty$ faster than $r^{-2}$ so that all weak field tests are still satisfied. A similar solution was also found earlier by \cite{1979GReGr..10...79B}. Working in Boyer-Linquist-like coordinates \cite{2011PhRvD..83j4027V} found a series of differential relations that must be satisfied by the metric perturbation components. The only non zero components of the perturbation are $\left\{ h_{tt}, h_{t \phi }, h_{rr}, h_{ \phi \phi } \right\}$; in particular $h_{\theta\theta}$ vanishes. In subsequent work \cite{2011PhRvD..84f4016G} expansions for the metric perturbation components in powers of $1/r$ were found,
\begin{equation} h_{\mu \nu}=\sum_{n}h_{\mu \nu, n}\left( \frac{1}{r} \right)^{n} \end{equation}
where the coefficients $h_{\mu \nu , n}$ are functions of $\theta$ only. The leading order non-zero coeffeicients are given in Eq.\ \ref{eq:B2metriccomps} and all the coefficients up to ${\cal{O}}(1/r^{5})$ are reproduced in Appendix \ref{subsec:BNcomponents};
\begin{eqnarray}\label{eq:B2metriccomps}
&h_{tt,2}&=\gamma_{1,2}+2\gamma_{4,2}-2a\gamma_{3,1}\sin^{2}\theta \, , \nonumber \\
&h_{rr,2}&=-\gamma_{1,2}\, ,\nonumber\\
&h_{t\phi,2}&=-M\sin^{2}\theta\left[ \gamma_{3,3}+a\left(\gamma_{1,2}+\gamma_{4,2} \right) +a^{2}\gamma_{3,1} \right] \, , \nonumber \\
&h_{\phi\phi,0}&=2M^{2}a\gamma_{3,1}\sin^{4}\theta\, .
\end{eqnarray}
Note that the leading order correction enters at a lower order in the $h_{\phi\phi}$ component. The components $h_{tt}$ and $h_{rr}$ are dimensionless, whilst $h_{t\phi}$ has units of length and $h_{\phi\phi}$ has units of length square; reflecting the dimensions of the components of $g_{\mu\nu}$.

At low order in the expansion the metric perturbation is fully characterised by a small number of coefficients, $\gamma_{i,j}$. Adopting the notation of \cite{2011PhRvD..84f4016G} the metric perturbation up to ${\cal{O}}(1/r^{2})$ is given by the four constants ${\cal{B}}_{2}=\left(\gamma_{1,2},\gamma_{3,1},\gamma_{3,3},\gamma_{4,2}\right)$ (see Eq.\ \ref{eq:B2metriccomps}); up to ${\cal{O}}(1/r^{3})$ it is given by the 7 constants ${\cal{B}}_{2}\cup{\cal{B}}_{3}$, where ${\cal{B}}_{3}=(\gamma_{1,3}, \gamma_{3,4}, \gamma_{4,3})$; up to ${\cal{O}}(1/r^{4})$ it is given by the 10 constants ${\cal{B}}_{2}\cup{\cal{B}}_{3}\cup{\cal{B}}_{4}$, where ${\cal{B}}_{4}=(\gamma_{1,4}, \gamma_{3,5}, \gamma_{4,4})$; up to
${\cal{O}}(1/r^{5})$ it is given by the 13 constants ${\cal{B}}_{2}\cup{\cal{B}}_{3}\cup{\cal{B}}_{4}\cup{\cal{B}}_{5}$, where ${\cal{B}}_{5}=(\gamma_{1,5}, \gamma_{4,5}, \gamma_{3,6})$.

Throughout this paper various ${\cal{B}}_{N}$ limits will be referred to, these correspond to setting all the parameters $\gamma_{i,j}$ to zero except for those in the set ${\cal{B}}_{N}$, which are treated as independent free parameters. This greatly reduces the number of free parameters and allows for a systematic way of examining perturbations to the Kerr metric. Because the Kerr metric is known to be an excellent approximation at large radii it is natural to expect any deviation from the Kerr solution to show up, initially at least, at lowest order in $1/r$. This is the justification for examining each of the ${\cal{B}}_{N}$ limits separately. In general, it would be possible for any combination of the $\gamma_{i,j}$ to be be none-zero. In this paper results are presented for the ${\cal{B}}_{2}$, ${\cal{B}}_{3}$, ${\cal{B}}_{4}$ and ${\cal{B}}_{5}$ metrics.

\subsection{Known black hole solutions}
In addition to the very general family of perturbed metrics described above, it is also useful to consider some examples of known black hole solutions in alternative theories of gravity. In this section a few such solutions are listed. One of these spacetimes (the linear in spin solution to dynamical Chern-Simons gravity, Sec.\ \ref{subsec:CS1}) is a member of the family of solutions described in Sec.~\ref{sec:spacetimes}; it can be obtained by a particular choice of the constants $\gamma_{i,j}$. 

\subsubsection{The Kehagias Sfetsos metric}\label{subsec:KS}
A spherically symmetric blackhole solution to Ho\v{r}ava gravity \citep{2009PhRvL.102p1301H,2009PhRvD..79h4008H} was found by \cite{2009PhLB..678..123K}. This was generalised to a slowly rotating solution by \cite{2010EPJC...70..367L}. Accretion disk signatures for this type of black hole have been considered previously by \cite{2011CQGra..28p5001H}. The metric is
\begin{eqnarray}\label{eq:KSmetric} 
&\textrm{d}s^{2}_{\textrm{KS}}&=-f_{\textrm{KS}}(r)\textrm{d}t^{2}+\frac{\textrm{d}r^{2}}{f_{\textrm{KS}}(r)}+r^{2}\textrm{d}\Omega^{2}-\frac{4a\sin^{2}\theta}{r}\textrm{d}\phi\textrm{d}t\; , \nonumber\\
&&\textrm{where }\, f_{\textrm{KS}}(r)=1+\omega r^{2}\left(1-\sqrt{1+\frac{4}{\omega r^{3}}}\right)\,, \nonumber \\
&&\textrm{and }\,\textrm{d}\Omega^{2}=\textrm{d}\theta^{2}+\sin^{2}\theta\textrm{d}\phi^{2}\, .\end{eqnarray}
In the limit $\omega \rightarrow \infty$ the slowly rotating limit of the Kerr metric is recovered. In order to avoid a naked singularity at the origin an extra constraint is needed, $\omega M^{2} \geq \frac{1}{2}$. For the remainder of this paper $\omega$ is made dimensionless by multiplying by $M^{2}$, and we choose to work with the small parameter $Y\equiv 1/(\omega M^{2})\ll 1$. For the remainder of this paper the metric in Eq.\ \ref{eq:KSmetric} will be refered to as the KS metric.


\subsubsection{The Chern Simons metric linear in spin}\label{subsec:CS1}
Dynamical Chern Simons modified gravity \cite{2003PhRvD..68j4012J} is a parity violating theory of gravity constructed by adding an Pontryagin invariant term to the action. As the Schwarzschild solution is spherically symmetric and has even parity it remains a solution to the modified CS field equations; however, the Kerr solution does not have even parity and fails satisfy these equations. No complete rotating black hole solution is known in the CS theory, however perturbative solutions in the spin and CS coupling constant have been found analytically. The rapidly rotating case was considered numerically by \cite{2014arXiv1407.2350S}, here we focus on the slowly rotating solutions.

The slowly rotating black hole solution (linear in spin) to dynamical Chern-Simons gravity was found by \cite{2009PhRvD..79h4043Y};
\begin{equation}\label{eq:CS1metric} \textrm{d}s^{2}_{\textrm{CS}1}=\textrm{d}s^{2}_{\textrm{Kerr}}+\frac{5}{4}\zeta\chi\frac{1}{r^{4}}\left(1+\frac{12}{7r^{2}}+\frac{27}{10r^{2}}\right)\textrm{d}t\textrm{d}\phi \, .\end{equation}
Accretion disk signatures for this type of black hole have been considered previously by \cite{2010CQGra..27j5010H}. For the remainder of this paper the metric in Eq.\ \ref{eq:CS1metric} will be refered to as the CS1 metric.

\subsubsection{The Chern Simons metric quadratic in spin}\label{subsec:CS2}
A slowly rotating black hole solution (quadratic in spin) to dynamical Chern-Simons gravity was found by \cite{2012PhRvD..86d4037Y}. As far as the authors are aware disk emission in this metric has not been considered before. The metric is
\begin{equation}\label{eq:CS2metric} \textrm{d}s^{2}_\CSt = \textrm{d}s^{2}_{\textrm{Kerr}}+\delta\left(g_{\mu\nu}^\CSt \right)\textrm{d}x^{\mu}\textrm{d}x^{\nu}\, \end{equation}
and expressions for the components $\delta\left(g_{\mu\nu}^\CSt \right)$ are given in Appendix \ref{subsec:CS2components}. For the remainder of this paper the metric in Eq.\ \ref{eq:CS2metric} will be referred to as the CS2 metric.

Unlike the KS metric, both the CS1 and CS2 metrics recover the Schwarzschild solution in the limit $a\rightarrow 0$. This is associated with the fact that dynamical Chern Simons modified gravity is a parity violating theory of gravity and hence the spherically symmetric Schwarzschild solution in GR, which is parity even, remains a solution of the modified theory.

