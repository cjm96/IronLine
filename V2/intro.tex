\section{Introduction}
General Relativity (GR) has been extensively tested in the weak field regime. The famous classical tests of GR began before the theory was even fully formulated with the explanation of the anomalous precession of the perihelion of mercury \cite{1915SPAW.......831E}. Only a few years later further confirmation came with the first measurements of the gravitational deflection of light during the 1919 solar eclipse \citep{1920RSPTA.220..291D}. More recently, but no less famously, the confirmation by \cite{1975ApJ...195L..51H} that the loss of energy from the binary pulsar PSR1913+16 was in the manner predicted by the theory won the authors the 1993 Nobel prize in physics~\footnote{\url{http://www.nobelprize.org/nobel_prizes/physics/laureates/1993/index.html}}. Interspersed among these landmark experiments GR has been the subject of constant stream of increasingly stringent tests; to date no deviations from the predictions of GR have been detected. For a review of experimental tests of GR see \cite{lrr-2006-3}. 

All of the tests mentioned above concern phenomena in weak gravitational fields; it has proved much more difficult to devise similarly stringent tests of GR in the strong gravitational field. This is due in part to the immense difficulty associated with measuring the motion of bodies on the small gravitational length scale associated with black holes (a few kilometers for a solar mass black hole) at astrophysical distances. If such measurements could be made and it were possible to accurately track the motion of a test particle around an astrophysical black hole it would become possible to place constraints on the strong field spacetime metric. It is a prediction of GR, together with the no-hair theorem, that the spacetime metric around an astrophysical black holes is described by the famous Kerr solution \citep{PhysRevLett.11.237}. Alternative theories of gravity predict the existence of different solutions (examples of known solutions considered in this paper include the Kehagias-Sfetsos solution \citep{2009PhLB..678..123K} to Ho\v{r}ava gravity \citep{2009PhRvL.102p1301H,2009PhRvD..79h4008H} and the slowly rotating solution \citep{2009PhRvD..79h4043Y} to dynamical Chern-Simons gravity \citep{2003PhRvD..68j4012J}). If a deviation from the Kerr solution was detected it could indicate a failure of GR in the strong field regime. Alternatively, it could indicate a failure of the no-hair theorem and raise the possibility of exotic compact objects within GR. It should be noted that there are other ways in which alternative theories of gravity may differ from GR besides changes to the metric around a black hole; for example, scalar-tensor theories of gravity (of which Brans-Dicke theory \citep{PhysRev.124.925} is perhaps the best known example) generically predict the existence of addition gravitational wave polarisation states \citep{lrr-2006-3}. In this paper however we restrict our attention to testing the spacetime metric around a black hole.

Tests of the metric around a black hole may be possible in the near future with gravitational wave observations, such as those from ground-based detectors like advanced LIGO and advanced VIRGO \citep{2010CQGra..27h4006H,Acernese2009}. For example, the possibility of using a network of ground-based gravitational wave detectors observing the coalescences of binary neutron stars to constrain the deviation of a post-Newtonian coefficient from the predicted GR value was considered in \cite{2014PhRvD..89h2001A}. Further into the future, low-frequency space-based gravitational wave observatories such as eLISA \citep{TheGravitationalUniverse} will offer the possibility of probing the gravitational field around supermassive black holes. The possibility of using observations of the inspiral of a stellar mass compact object into a supermassive black hole to constrain the anomolous quadrupole moment of the massive black hole was considered by \cite{1997PhRvD..56.1845R}, and later by \cite{2007PhRvD..75d2003B}. These gravitational wave observations are ideally suited to such tests as they offer the ability to accurately track the orbital phase of the two bodies all the way up to and including the merger. However, as gravitational wave observation are not yet available it is interesting to ask if strong field tests are possible using current electromagnetic observations.  

Accretion disks provide an ideal candidate for such tests, because X-ray irradiation of matter in the inner regions of the disk ($r \lesssim 20M$) imprints characteristic features upon the X-ray spectra, in particular the fluorescent K$\alpha$ Iron line (rest energy 6.38keV). As this emission originates from so close to the Black Hole (BH)  it is strongly distorted by gravitational and Doppler shifting producing the characteristic `two-horned' profiles observed in nature \cite{1995Natur.375..659T}. Furthermore the light, once emitted, is strongly gravitationally lensed as it propagates through the spacetime, altering the observed spectra and imprinting upon it extra information about the strong field metric. Indeed the large width of these lines is evidence for the existence of highly spinning black holes \citep{1996MNRAS.279..837I}; for a recent review of black hole spin measurements using X-ray emission see \cite{2011CQGra..28k4009M}. In addition, viscous torques present in the disk dissipate energy causing the material to gradually spiral inwards, this energy is radiated locally in the form of thermal emission. This thermal radiation is subjected to the same strong-field gravitational effects as the line emission, and so may also be used as a probe of the metric. 

The approach of this paper is to consider Iron line and thermal emission in a large class of parametrically deformed Kerr black holes. The deformed black hole spacetimes are referred to as ``bumpy black holes'' and the individual deformation parameters as ``bumps''. The spacetimes all have the property that if all the bump parameters are simultaneously set to zero then the Kerr solution is recovered. The advantage of this approach, compared with considering the emission in a small number of known alternative black hole solutions, is that one is able to consider a wide range of different deformations simultaneously and identify particular ``bumps'' which are especially easy or difficult to constrain. Furthermore, even if a particular alternative black hole is not contained within this class it may be hoped that the disk spectrum in this new metric would have a significant overlap with the spectrum of a black hole in the class; and therefore a deviation from GR would still be detectable. The family of bumpy black holes used for this purpose was the class of metrics constructed by \cite{2011PhRvD..83j4027V}; these have the property that the perturbed spacetime possess a fourth constant of motion (analogous to the Carter constant \citep{PhysRev.174.1559} in Kerr spacetime). In addition to this very general familiy of bumpy black holes the results are put into context by comparing them with the bounds it is possible to place on some known black hole solutions; in particular both the linear \cite{2009PhRvD..79h4043Y} and quadratic \citep{2012PhRvD..86d4037Y} in spin solutions to weakly coupled dynamical Chern-Simons graity, and the slowly rotating Kehagias-Sfetsos black hole \citep{2010EPJC...70..367L} were considered.

In this paper we begin in Sec.\ \ref{sec:spacetimes} by describing the family of bumpy black holes that were considered in this paper, we also describe a couple of specific examples of known black hole solutions in alternative theories of gravity which are also considered here. In Sec.\ \ref{sec:emission} the necessary theory for calculating both the thermal and Iron line emission from an accretion disk in a general stationary, axisymmetric spacetime is described. Sec.\ \ref{sec:analysis} describes the methods used for data analysis is this paper and the method used for estimating the bounds it will be possible to place of the different deformation, or ``bump'', parameters. The results are presented in Sec.\ \ref{sec:results} and finally a discussion and concluding remarks are given in Sec.\ \ref{sec:discussion}. Throughout this paper natural units, where $G=c=k_{B}=1$, are used.


